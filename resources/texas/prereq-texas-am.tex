% Syracuse 2012

\documentclass[10pt,a4paper]{article}
\usepackage[utf8]{inputenc}
\usepackage[charter]{mathdesign}
\usepackage[scaled=.95]{inconsolata}
\usepackage[margin=1.2in]{geometry}
\usepackage{color}

\usepackage{hyperref}
\usepackage{dcolumn,booktabs} %% for memisc
\usepackage{graphicx}
\usepackage{amsmath}

\usepackage{stitchr} %% make me compile as latex when knitr is not applied

\definecolor{darkblue}{rgb}{0,0,.6} % not really
\definecolor{other}{rgb}{0,0,.5}
\hypersetup{colorlinks=true, linkcolor=darkblue, citecolor=darkblue, 
	filecolor=darkblue,urlcolor=other}

\setlength{\parskip}{1em}
\setlength{\parindent}{0em}


\author{Texas A\&M 2014}
\title{Prerequisites: Software}

\date{}

\begin{document}
\maketitle

We'll be using some software for this course.  It's all free, but you will want to download and install it.

\begin{itemize}
\item \textbf{Java}: Press the big red button at \url{http://www.java.com/en/} and follow the instructions.  If you are a Mac user you already have Java installed, although you may need to upgrade to get the latest version.  This will require administrator access on your computer. 
\item \textbf{R}: Download by choosing a nearby location from \url{http://cran.r-project.org/mirrors.html} and selecting the distribution appropriate to your computer.  This may require administrator access to your computer.

Mac and Linux users should have no problems with the procedure, although it may require an administrator password.  

Windows users can have a harder time.  The following notes apply to them.  Let us start with a quote from the FAQ for R on Windows (\url{http://cran.r-project.org/bin/windows/base/rw-FAQ.html})
\begin{quote}
To install use `R-2.15.1-win.exe'. Just double-click on the icon and follow the instructions. If you have an account with Administrator privileges you will be able to install R in the Program Files area and to set all the optional registry entries; otherwise you will only be able to install R in your own file area. You may need to confirm that you want to proceed with installing a program from an `unknown' or `unidentified' publisher.
\end{quote}
Two particular notes:  If you get to any point during installation where you are given the option to install with `internet' or `internet2' options, choose the latter.  This will take internet settings from your browser.  Assuming this is set up right, this will guarantee internet access.  In theory this is no longer a concern. In practice\ldots

\textbf{Testing 1, 2, 3} Launch R from the Desktop link (Windows) or otherwise.  Then type
\begin{quote}
\texttt{install.packages('austin', repos='http://r-forge.r-project.org')}
\end{quote}
If this succeeds, perhaps after requesting a mirror site from a list, then R is working properly.  If not, then re-installing is probably necessary.
\item \textbf{R Packages: austin, RTextTools, and lda}.  Install these by selecting them from the install packages menu individually, or by typing, e.g. 
\begin{quote}
\texttt{install.packages('RTextTools')}
\end{quote}
at the prompt. 

If you plan to make sweet music with R software in your academic career then you might also like RStudio \url{http://rstudio.org/download/desktop} as an alternative environment for using R once it has been installed.  We won't make use of this here.
\end{itemize}


\end{document}

